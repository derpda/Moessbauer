\section{Summary}
\subsection{Background}
By measuring the absorption for different thicknesses of aluminum, the background count rate was determined to be:
\begin{equation*}
A_C= (29.74\pm0.17)s^-1
\end{equation*}	
\subsection{Attenuation by the acrylic glass}
The attenuation coefficient for the acrylic glass was measured with the result:
\begin{equation*}
\mu_{acr}=\unit{(1.21\pm0.02)}{1/cm}
\end{equation*}
The value does not match the calculated one $\mu=\unit{1.31019}{1/cm}$. For the latter, nominal values for $E_\gamma$=\unit{15}{keV} were used, which is slightly higher then the energy of the iron transition. The effect of this would be lower attenuation coefficients however. Their difference can thus only be explained by different thicknesses of the glass for the absorber and the empty acrylic glass plate. A $5\%$ difference in thickness would close the gap between the values.
\subsection{Single line absorber}
The isomeric shift was measured
\begin{equation*}
E_{iso}= \unit{8.61\pm0.24\cdot10^{-9}}{eV}
\end{equation*}. 
With the calculated effective absorber thickness $T_A=(6.2\pm0.4)$, the Debye-Waller factor of the source was determined:
\begin{equation*}
f_s=0.51\pm0.08
\end{equation*}
The mean life as determined from the Voigt fit is:
\begin{equation*}
\scalebox{1.5}{$\tau$}=\unit{(195\pm140)}{ns}
\end{equation*}
Although it lies within one standard deviation to the literature value scalebox{1.5}{$\tau$}$_{lit}$=\unit{140}{ns}. This only due to the large uncertainty (over $70\%$) caused by the relative error of the $\delta$ parameter of the Voigt fit ($70\%$). With the second method, via the effective absorber thickness, a life time of
\begin{equation*}
\scalebox{1.5}{$\tau$}= \unit{(170\pm40)}{ns}
\end{equation*}
was determined.
\subsection{Six line absorber}
By fitting the sum of six Lorentz curves to the recorded spectrum of natural iron, the following values were determined during evaluation.
\begin{table}[H]\centering
	\begin{tabular}{@{}llllll@{}}
		\toprule
		 Quantity& Value \\
		\midrule
		$\overline{E_{iso}}$ & \unit{(4.9\pm1.1)}{neV}\\
		$\mu_ea$ & \unit{(-0.146\pm0.005)}{\mu_N} \\
		$\mu_eb$ & \unit{(-0.162\pm0.003)}{\mu_N}\\
		$\mu_ec$ & \unit{(-160\pm0.003)}{\mu_N}\\
		$B_a$ & \unit{(33.7\pm0.7)}{T}  \\
		$B_b$ & \unit{(32.4\pm0.4)}{T}  \\
		$B_c$ & \unit{(31.7\pm1.5)}{T}  \\ 
		\bottomrule
	\end{tabular}
	\caption[Summary six line absorber]{Summary of the results for the six line absorber}
	\label{tb:summary:sixline abs}
\end{table}
The isomeric shift lies within one standard deviation at the literature $E^{lit}_{iso}=\unit{5.3}{neV}$ value. The measured values for the magnetic moment enclose the literature  value $\mu_e (-0.1549\pm0.002)$\cite{stone} within their 3$\sigma$ intervals. The measured values of the magnetic field overlap with literature value $B_{lit}=\unit{35}{T}$

