\section{Summary}
\subsection{Calibration}

\subsection{Background}
By measuring the absorption for different thicknesses of aluminum, the background count rate was determined to be:
\begin{equation*}
A_C= (29.74\pm0.17)s^-1
\end{equation*}	
\subsection{Attenuation by the acrylic glass}

\subsection{Single line absorber}
The isomeric shift was measured\begin{equation*}
E_{iso}= \unit{8.61\pm0.24\cdot10^{-9}}{eV}
\end{equation*}. 
With the calculated effective absorber thickness $T_A=(6.2\pm0.4)$ the Debye-Waller factor of the source was determined:
\begin{equation*}
f_s=0.51\pm0.08
\end{equation*}
Both measurements of the mean life time of the \unit{14.4}{keV} state, namely \scalebox{1.5}{$\tau$}=\unit{(11\pm8)}{ns} (directly from a fit parameter) and \scalebox{1.5}{$\tau$}=\unit{(6.5\pm0.4)}{ns} are far lower than the literature value \scalebox{1.5}{$\tau$}$_{lit}$ = \unit{140}{ns}.

\subsection{Six line absorber}
By fitting the sum of six Lorentz curves to the recorded spectrum of natural iron, the following values were determined:
\begin{table}[H]\centering
	\begin{tabular}{@{}llllll@{}}
		\toprule
		 quantity& value \\
		\midrule
		$\overline{E_{iso}}$ & \unit{(4.9\pm1.1)}{neV}\\
		$\mu_ea$ & \unit{(0.146\pm0.005)}{\mu_N} \\
		$\mu_eb$ & \unit{(0.162\pm0.003)}{\mu_N}\\
		$\mu_ec$ & \unit{160\pm0.003}{\mu_N}\\
		$B_a$ & \unit{33.7\pm0.7}{T}  \\
		$B_b$ & \unit{32.4\pm0.4}{T}  \\
		$B_c$ & \unit{31.7\pm1.5}{T}  \\ 
		\bottomrule
	\end{tabular}
	\caption[summary six line absorber]{Summary of the results for the six line absorber}
	\label{tb:summary:sixline abs}
\end{table}
