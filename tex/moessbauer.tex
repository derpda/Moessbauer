\documentclass[11pt, paper=a4]{scrartcl}

% CONFIG
\newcommand{\exptitle}{M\"o\ss bauer effect}       % long name of experiment 
\newcommand{\exptitleshort}{M\"o\ss bauer effect} % short name of experiment
\newcommand{\expdate}{04.04.2016}           % date of experiment
\newcommand{\exptutor}{Veronika Magerl}

\input{Vorspann}


\pagenumbering{Roman}
\setcounter{page}{1}


%\section*{Abstract}
In the experiment the absorption spectrum of stainless steel and natural iron are recored by using the Mössbauer effect. The Gamma radiation of a Fe-57 decay is Doppler shifted by moving the absorber relative to the source. This allows for a high relative resolution, enabling the measurement of the isometric shift in stainless steel $E_{iso}= \unit{8.61\pm0.24\cdot10^{-9}}{eV}$ and natural iron $\overline{E_{iso}} = \unit{(4.9\pm1.1)}{neV}$. Furthermore the Debye-Waller factor was calculated to be $f_s=0.51\pm0.08$.
\tableofcontents

\newpage
\listoffigures
\thispagestyle{empty}

\newpage
\hypersetup{pageanchor=false} %stop page numbering (hyperref) to prevent for double page numers

\clearpage
\pagenumbering{arabic}
\setcounter{page}{1}

%physical principles

\section{physical principles}
\begin{figure}[H]
	\centering
	\includegraphics[height=0.18\textheight]{graphics/Emission}
	\caption[spontaneous $\gamma$ emission]{principle of spontaneous $\gamma$ emission of excited nuclei. Transitioning from an excited state ($E_a$) to the ground state ($E_g$) the nucleus emits a photon with energy $E_a-E_g=\hbar\dot \omega$ or transmits that energy directly to an electron of the atomic shell.\cite{Wegener}}
	\label{fig:principles:Emission}
\end{figure}
\subsection{Gamma Decay and and resonance Absorption}
Nuclei in excited states (energy $E_a$) can spontaneously transition into the ground (energy $E_g$) state. The energy difference $\Delta E=E_a-Eg$ is either directly gained by a shell electron (inner conversion), or carried by a emitted photon (spontaneous emission). The frequency $\omega$ of the photon is given by: 
\begin{equation}
\Delta E = h\cdot \omega
\end{equation}
where $\hbar = 6.582 119 514 \cdot 10^{-16} eVs$ is the Planck constant over $2\pi$ \cite{Codata}.
The Reverse process is also possible, namely a nucleus can transit into an excited state by absorbing a photon.

However the emission (and absorption) spectrum is not infinitely sharp, but a Lorentz distribution with natural line width $\Gamma$ (see figure \ref{fig:principles:lorentz}) \cite{Wegener}: 
\begin{equation}
I(\omega) \propto \frac{1}{(\omega-\omega_0)^2+(\Gamma /2)^2}
\label{eq:principles:Lorentz}
\end{equation}
\begin{figure}[H]
	\centering
	\includegraphics[height=0.25\textheight]{graphics/Lorentz.pdf}
	\caption[Lorentz distribution]{Illustration of a Lorentzian curve}
	\label{fig:principles:lorentz}
\end{figure}

 The line width is related to the mean life-time \scalebox{1.5}{$\tau$} of the excited state via Heisenberg's uncertainty relation (here energy-time uncertainty):
 
 \begin{equation}
 \hbar=\Gamma \cdot \scalebox{1.5}{$\tau$}
 \end{equation}

\subsection{Interaction of Gamma radiation with matter}
Photons interact with matter in three major ways\cite{Demtröder}:

\paragraph{Photoelectric effect} \ \\
Shell electrons of atoms absorb photons and gain its energy, leaving the potential well of the atom and exiting the shell with the energy $E_e = E_\gamma-E_B$ with $E_B$ being the binding energy of the electron.
\paragraph{Compton scattering} \ \\
Compton Scattering is the elastic scattering of photons at quasi free  electrons ($E_B << E_\gamma$) and its wavelength $\lambda=2\pi c/\omega$ shifted, depending on the scattering angle $\varphi$(see figure \ref*{eq:compton}):
\begin{equation}
\lambda_S -\lambda_0 = \frac{2 \pi \hbar}{m_e c}(1-cos(\varphi))
\label{eq:compton}
\end{equation}
\begin{figure}[h]
\centering
\includegraphics[width=0.5\linewidth]{graphics/Compton}
\caption{Compton effect: A photon is scattered by a (quasi) free electron changing its direction by an angle $\varphi$ }
\label{fig:principles:Compton}
\end{figure}

\subsection{Moessbauer effect}


\begin{figure}[hb]
	\centering
	\includegraphics[width=1.0\linewidth]{graphics/Zerfallsschema}
	\caption[Co-57 decay]{decay series of Cobalt-57}
	\label{fig:principles:Zerfallsschema}
\end{figure}

\section{Experimental setup and procedure}
\subsection{Method}
To measure the absorption spectra of stainless steel and natural iron, we irradiate the samples with the $14.4keV$ $\gamma$-radiation emitted by a radioactive source. To vary the frequency, a motor is used to move the absorber relative to the source (Doppler shift see eq.  \ref{eq:diffdopplershift}). By repeating this measurement for different absorber velocities, a spectrum is recorded.\\
\subsection{Setup}
\begin{figure}[hbt]
\centering
\includegraphics[width=1.0\linewidth]{graphics/Aufbau}
\caption[Overview of the experimental set-up]{Overview of the experimental set-up. The motor control unit can be controlled from the computer. \cite{anleitung}}
\label{fig:Aufbau}
\end{figure}

The setup consists of the $\gamma$ source, the absorber on a track, the motor used to move the absorber at constant speeds relative to the source and as the photon detector, a scintillator is used. The light signal of the scintillator is turned into an electric signal by a photomultiplier. This signal is amplified and shaped in the amplifier. The amplifier has two exits, one of which is connected to a single channel analyzer (SCA). If the signal pulse is within an adjustable window, the SCA sends a standardized signal and enables the linear gate, which is also connected to the amplifier via a delay to ensure simultaneity of the signals. If the linear gate is enabled when it receives a signal from the amplifier it transmits the amplifier signal to the multichannel analyzer (MCA), which is read out with a computer. The second output of the SCA is connected to a counter, which also can be read out with the Computer.

\subsection{The source Co-57}
\isotope[57]{Co} decays via electron capture with a branching ratio of $99.8 \%$ and a half life of 270d into a iron in an excited state $\isotope[57]{Fe^*}$. This state decays with a half life of 9ns and a branching ratio of $88\%$ into the $14.4$keV excited state, which finally decays to the ground state (Branching ratio for $\gamma-decay$ is $10\%$) \ref{fig:principles:Zerfallsschema2}.
\begin{figure}[hbt]
	\centering
	\includegraphics[width=0.5\linewidth]{graphics/Zerfallsschema2}
	\caption[Co-57 decay]{Decay series of Cobalt-57. \cite{khwz}}
	\label{fig:principles:Zerfallsschema2}
\end{figure}
\subsection{Americium sample}
To calibrate the MCA, multiple reference samples are used. For this purpose, americium is used as a primary source. The americium is shielded in an stainless steel case with an aperture. Attached in front of this aperture is a disc with multiple targets (Cu, Rb, Mo, Ag, Ba, and Tb), accessible by rotating the disc (see fig \ref{fig:Americium}). The radiation of the Americium source excites the target material which in turn starts emitting characteristic x-rays (x-ray fluorescence)\cite{landgraf}. The characteristic lines of the targt samples can be found in fig. \ref{fig:Americium_table} and for the americium source in fig. \ref{fig:Amercium_lines}.
\begin{figure}[h]
\centering
\includegraphics[width=0.7\linewidth]{graphics/Americium}
\caption[Americium sample]{Americium sample with target revolver used as  reference for the MCA calibration. \cite{anleitung}}
\label{fig:Americium}
\end{figure}
\begin{figure}[H]
\centering
\includegraphics[height=0.2\textheight]{graphics/Americium_table}
\caption[Line data for the calibration samples]{$K_\alpha$ and $K_\beta$ lines of Cu, Rb, Mo, Ag, Ba, and Tb. \cite{anleitung}}
\label{fig:Americium_table}
\end{figure}
\begin{figure}[H]
\centering
\includegraphics[height=0.2\textheight]{graphics/Amercium_lines}
\caption[5 visible Americium lines]{The 5 visible lines of the americium source. \cite{landgraf}}
\label{fig:Amercium_lines}
\end{figure}

\subsection{Procedure}
\subsubsection{MCA calibration}
First the window size was set to maximum and the spectrum of the cobalt source was recorded.
To identify the \unit{14.4}{keV} peak of the source, the (known) spectra of Cu, Rb, Mo, Ag, Ba, and Tb are measured for 300s each. The results are used to identify the \unit{14.4}{keV}vin the source spectrum. The Window of the SCA was adjusted accordingly, by recording the source spectrum while adjusting the window and repeatedly resetting the recording on the computer. The Window was then adjusted until only the channels of the \unit{14.4}{keV} peak get a signal. We chose the settings:
\begin{itemize}
	\item upper level: 1.10
	\item lower level: 0.69
\end{itemize}

\subsubsection{background measurement}
The main source of background are photons of the transition between the \unit{136.5}{keV} state and the \unit{14.4}{keV} state (see fig \ref{fig:principles:Zerfallsschema2}) being scatter via Compton scattering in the scintillator and falling in the adjusted SCA window. To measure this background, aluminum plates of different thicknesses (measured with) are used to shield the scintillator. For each plate the event counts were measured over 600s. The plate thicknesses were measured with a caliber. 

\subsubsection{Absorption spectra of stainless steel}
First a rough measurement is made: the absorption was measured for velocities of 0.1 mm/s to 1.1 mm/s (both directions) in steps of 0.1mm/s for 180s. For the finer measurements a measuring time of 300s was chosen.
\subsubsection{Absorption spectra of natural iron}
For natural iron, the absorption was measured for absorber speeds between 0.1mm/s and 8mm/s in steps of 0.1mm/s. In a second measurement the range 0.05mm/s to 6.05mm/s was taken, also in steps of 0.1mm/s. The measuring for each velocity was 300s.
\subsubsection{Attenuation through acrylic glass}
The absorber is removed from the setup and the counting rate measured for 900s, once with acrylic glass and once without.

\newpage
\begin{thebibliography}{9}
\bibitem{Wegener}
Wegener, Horst. "Der Mößbauer Effekt und seine Anwendungen". Mannheim 1966
\bibitem{Demtröder}
Demtröder, Wolfgang. Experimentalphysik 3 Atome, Moleküle und Festkörper
\bibitem{Codata}
P.J.Mohr, D.B.Newell, and B.N. Taylor:\\ "CODATA Recommended Values of the Fundamental Physical Constants: 2014". http://physics.nist.gov/cuu/Constants/index.html(26.04.2016)
\bibitem{jakobs}
Jakobs, Karl. Experimentelle Methoden der Teilchenphysik. Vorlesungsskript 2014
\bibitem{Eyges}
Eyges, Leonard. Physics of the Mössbauer effect. 1965
\bibitem{anleitung}
A.Zwerger(2007), S.Winkelmann(1/2011), M.Köhli (2/2011). Versuchsanleitung Fortgeschrittenen Praktikum Teil II - Der Mößbauer-Effekt. 2012
\bibitem{khwz}
M.Köhler(8/2010), M.köhli (4/2011). Versuchanleitung Fortgeschrittenen Praktikum Teil I - Kurze Halbwertszeiten
\end{thebibliography}



\end{document}
